\section{商品信息爬取}

我们针对苏宁(https://www.suning.com)中的数码产品信息,总共爬取了3984个商品页面,获取其中的商品名称、页面名称、图片url、品牌等信息。爬虫分为三部分:爬取商品价格、爬取商品基本信息、爬取商品评价Ajax请求url地址。爬虫的架构以及网页信息保存格式参考了上机实验5的网页爬虫以及图片爬虫。

\subsection{爬取商品基本信息}

商品的基本信息包括了网页标题(title)、网页url(url)、商品图片url(imgurl)、商品名称(name)、品牌(brand)、价格(price)、类别(attribute)。

这些信息可以直接在网页源码中找到,用beautifulsoup获取。对于不同商品,图片url地址、商品名称、品牌以及类别信息在网页源码中的位置是固定的,通过查看网页源码可以确定各信息所在的标签的位置。

\begin{python}
def get_data(content,url,price): # content:网页源码,url:网页url地址,price:商品价格
    soup = BeautifulSoup(content, features='html.parser')
    data = dict()
    imgurl = soup.findAll('a',{'id':'bigImg'})[0]
    imgurl = complete_url(imgurl.findAll('img')[0].get('src'))
    name_tag = soup.findAll('span', {'class':'breadcrumb-title'})[0]
    name = name_tag.get('title')
    dropdown = soup.findAll('div',{'class':'dropdown'})
    brand_tag = dropdown[-1].find('a')
    attribute_tag = dropdown[-2].find('a')

    data['title'] = soup.head.title.text
    data['url'] = url
    data['imgurl'] = imgurl
    data['brand'] = brand_tag.find_all(text=True)[0]
    data['name'] = name
    data['attribute'] = attribute_tag.text
    data['price'] = price
    return data
\end{python}

\subsection{爬取商品价格}

我们使用PySide库中的Webkit渲染引擎,在读取网页时会加载所有的动态页面和静态页面。Pyside.QtWebKit读取网页不能多线程进行,因此对苏宁的爬虫采用单线程方式。

\begin{python}
class BrowserRender(QWebView):
    def __init__(self, show=True):
        try:
            self.app = QApplication(sys.argv)
        except:
            self.app = QCoreApplication.instance()
        QWebView.__init__(self)
        if show:
            self.show()

    def download(self, url, timeout=60): # 读取页面
        loop = QEventLoop()
        timer = QTimer()
        timer.setSingleShot(True)
        timer.timeout.connect(loop.quit)
        self.loadFinished.connect(loop.quit)
        self.load(QUrl(url))
        timer.start(timeout * 1000)
        loop.exec_()
        if timer.isActive():
            timer.stop()
            return self.html()
        else:
            print "Request time out: " + url

    def find(self, pattern): # 查找页面中信息
       return self.page().mainFrame().findAllElements(pattern)

try:
    br.download(page)
    price = br.find("span.mainprice") # 商品价格在"span.mainprice"类别中
    price = price[0].toPlainText().encode("utf-8").strip()[2:] # 提取商品价格的数据
except:
    print "price error"
    continue
\end{python}

\subsection{爬取商品评论Ajax请求的url地址}

苏宁的商品评论以及评价标签通过JavaScript来加载数据,无法在网页源代码中找到,因此我们通过商品评论的Ajax请求地址直接获得商品的评价标签。

\begin{figure}[htbp]
\centering
\includegraphics[width=13.5cm]{img/fsh/1.jpg}
\caption{商品标签Ajax请求的URL}
\label{fig:fsh_crawler}
\end{figure}

苏宁的商品评价标签的Ajax请求地址的URL如图\ref{fig:fsh_crawler}所示,url中“30193816-10597918588-0000000000”对应了特定商品的评价标签。通过对网页源代码的查找,我们确定了这串数字为“cluster_id - commodity_id - store_id”的组织形式,cluster_id可以在网页源代码中找到,而commodity_id(商品编号)和store_id(店铺编号)可以在商品页面的url中直接获得。


\begin{python}
def get_comment_id(url,content):
    ret = urlparse.urlparse(url) # 使用urlparse解析url地址,获得提取url中的店铺编号和商品编号
    store_id = ret.path[1:11]
    commodity_id = ret.path[12:-5]
    content = get_page(url)
    x = content.find("clusterId") # clusterId位于js脚本中,无法通过beautiful获取,用字符方式在网页源码中搜索
    cluster_id = content[x+11:x+24]
    cluster_id = filter(str.isdigit, cluster_id)
    return '-'.join([cluster_id,commodity_id,store_id])
\end{python}