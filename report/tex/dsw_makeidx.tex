\chapter{构建索引}
在获取了商品属性、用户评价、商品打分等内容并分别将其储存在文本文件中后,我们接下来就可以利用Lucene建立索引了。由于京东和苏宁的商品信息都是这三类并且以相同的格式存储,我们接下来以京东为例解释建立索引的过程。

与本学期第五次试验建立Lucene索引的过程类似,首先我们打开一个储存有所有商品网址和文件名的文本,逐行读取商品URL和文件名,并从detail、 comment、score三个文件夹中找到储存该网页对应信息的三个文本。分别打开这三个文本文件,读取其中内容并进行合适的格式处理,最终添加为field。

我们的fieldType配置如下表所示。
\begin{table}[h]
\begin{tabular}{lccc}
\hline
\multicolumn{1}{c}{\textbf{FieldName}} & \textbf{Indexed} & \textbf{Stored} & \textbf{Tokenized} \\ \hline
URL                                    & T                & T               & F                  \\
price 商品价格                             & \multicolumn{3}{c}{built-in LONG Type}                  \\
score 商品评分                             & \multicolumn{3}{c}{built-in LONG Type}                  \\
imgurl 商品图片地址                          & F                & T               & F                  \\
title 商品名称                             & T                & T               & T                  \\
brand 商品品牌                             & T                & T               & T                  \\
attribute 商品类别                         & T                & T               & F                  \\
detail 商品详情                            & F                & T               & F                  \\
tag 商品标签                               & F                & T               & F                  \\
website 商品来源                           & F                & T               & F                  \\ \hline
\end{tabular}
\end{table}


首先注意到,价格“price”和打分“score”。前者从保存“detail”的文本中获取,后者则专门保存在“score”的文本中。由于这两样内容需要进行后续的数值比较,我们不能将其与其他field一样以“str” 形式建立,而要将之转换为long的格式,并建立longfield。以“score”为例:

\begin{python}
path3="new/score/" + filename2 + ".txt"
ff3 = open(path3)
line = ff3.readline().strip()
score = int(line.split('\t')[1])        #截取score
scoree = long(score)                    #转换为long型数
doc.add(LongField('score', scoree, Field.Store.YES)) #建立longfield,设为可存储
\end{python}


其次,商品标题(即商品简述)“title”。为了后续的检索,我们这里需要利用SimpleAnalyzer以及jieba分词库对其进行分词处理,并允许indexed、stored、tokenized。

此外,商品品牌、商品类别,由于是每个商品共有的属性,而且在多字段查询中也会被检索,因此我们也进行了索引或分词。

苏宁商品信息的索引建立过程与京东类似。至此,Lucene索引就建立完毕了。具体代码见文件“makeindex.py”。
