\section{网页URL爬取}

制作一个商品检索网站,第一步要收集足量的商品信息。我们计划从京东和苏宁两家电商上爬取足量的数码类商品,来完成本次大作业。经过信息的分析整理,进行分类,打分,贴标签等信息的提取和完善,完成一个可用的,好用的商品检索网站。

由于大作业是小组协作,先爬取1w+的商品,得到一个记录商品url的index.txt,后面的一切工作都围绕这些商品展开。这个爬虫主要沿用的是第三次上机中完成的带banlist的多线程爬虫。以爬取京东的爬虫为例,相对早期代码,主要的改变主要是限制了爬取范围在形如“https://item.jd.com/\{\}.html.format(productID)”的url——我们只需要商品详情页的信息。体现在代码中就是改变了获取链接的格式:

\begin{python}
def get_all_links(content):
    if content == None:
        return []
    import re
    urlset = set()
    urls = re.findall(r"item.jd.com/[\^\\s]*.html",content,re.I)
    for u in urls:
        urlset.add('http://'+u)
    links = list(urlset)
    return links

\end{python}

此外,爬取的起点也要是商品详情页。

完成了爬取之后,我们还需要进行细节信息的提取,评论信息的提取和打分,以完成整个信息准备的工作。下面我们将对这些环节进行详细介绍。


